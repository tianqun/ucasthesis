%---------------------------------------------------------------------------%
%->> Backmatter
%---------------------------------------------------------------------------%
\chapter{作者简历及攻读学位期间发表的学术论文与研究成果}


\section*{作者简历}

2018年9月 - 2019年9月~普渡大学地球、大气与行星科学系~联合培养

2014年9月至今~中国科学院大气物理研究所~气象学~硕博连读

2010年9月 - 2014年6月~中国海洋大学海洋与大气科学学院~大气科学~学士


\section*{已发表(或正式接受)的学术论文:}

{
\setlist[enumerate]{}% restore default behavior
\begin{enumerate}[nosep]
    \item MIAO H, DONG D, HUANG G, HU K, \textbf{TIAN Q}, et al. Evaluation of northern hemisphere surface wind speed and wind power density in multiple reanalysis datasets[J]. Energy, 2020:117382.
    
    \item \textbf{TIAN Q}, HUANG G, HU K, et al. Observed and global climate model based changes in wind power potential over the northern hemisphere during 1979–2016[J]. Energy, 2019, 167:1224-1235.
    
    \item XIE Z, DUAN A, \textbf{TIAN Q}. Weighted composite analysis and its application: an example using enso and geopotential height[J]. Atmospheric Science Letters, 2017, 18(11):435-440.
\end{enumerate}
}


\section*{参加的研究项目及获奖情况:}

国家杰出青年基金项目“热带海气相互作用及东亚季风系统”,项目编号: NSFC41425019.

\chapter[致谢]{致\quad 谢}\chaptermark{致\quad 谢}% syntax: \chapter[目录]{标题}\chaptermark{页眉}
\thispagestyle{noheaderstyle}% 如果需要移除当前页的页眉
%\pagestyle{noheaderstyle}% 如果需要移除整章的页眉

回想起6年前,满怀着憧憬来到大气物理研究所,一晃就到了毕业的时节,感概万千。

入学那年是国科大雁栖湖校区刚刚启用,离城区很远但紧邻雁栖湖景区,风景如画,在那里认识了一批很好的同学和朋友一起学习和玩耍,谢志昂、杨瑞、李普曦、李矜霄、韩永秋、鄢钰函、侯兆禄、薛佳庆、赵荐、周白羽、张洁、李飞、徐丹卉、杜惠云、范怡、雷婷、于水、赵宣铭、祝传栋、付远、韩韬、韦雯雯、江奇达、刘映雪、张天宇、袁善锋、王浩等等,也有幸聆听了一大批优秀学者讲授的课程,如,丁一汇院士、周天军研究员、陈文研究员、张井勇研究员、刘海龙研究员、俞永强研究员、王斌研究员、董理副研究员、严中伟研究员、林一骅研究员、Heki教授等等,感谢他们让我度过了快乐且有收获的研一时光。

回所之后,我的导师黄刚研究员和胡开明副研究员言传身教,让我在专业知识上不断提高,也逐渐了解了如何做研究,期间我也受到了大气物理研究所吴仁广研究员、李曦晨研究员、屈侠副研究员、王林副研究员、王鹏飞高工以及UCSD谢尚平教授、夏威夷大学金飞飞教授等的帮助和指导。另外研究所的孙鹏宇老师、付建建老师、刘洪涛老师、张予老师,课题组的师兄师姐陶炜晨、董丹红、王志彪、姜文萍、赵桂洁、赵文灿、朱丽华、黄勇、刘波、胡莉梭,师弟师妹王素、唐颢苏、苗昊泽宇、甘如玉、汪亚、周春江、李思萱、马晓帆、王秋琳、侯虹宇、周世杰,以及研究所内其他课题组的师兄师姐、师弟师妹和同学,杨耀先、陈东为、蒋如斌、李凯、李牧原、李文韬、李亚飞、李逸文、刘博、刘瑞金、刘森锋、彭冬冬、彭玉琢、陆婷婷、申冬冬、黄丽君、沈子力、黎慧琦、吴凡等等对我学业和生活上的帮助也让我倍感温暖。感谢他们的帮助让我能够顺利完成学业。

2018年,我获得了国家留学基金委公派的机会去美国普渡大学联合培养,因为这个契机我认识了的爱人罗茜博士,感谢她的出现让我每天都感到幸福,让我成为一个更好的人。在普渡大学期间,我有幸受到Dev Niyogi教授的指导,并认识了陈伯铭、Samuel Fung、周沛恩、申英男、陈静秋、潘峰、王淑媛、张帆、刘佳凯、Sajad、Pratiman、Nadu、Jie Liu、Alka等等一大批同学和朋友,感谢他们让我不虚此行。

感谢我的父母、家人、朋友对我一直的关心和支持,感谢中国科学院大学、大气物理研究所、普渡大学让我能有这么好的受教育的机会。希望我能够把我的所学用于服务大众,不管将来在学界还是业界都能为社会的进步贡献一份力量。


\cleardoublepage[plain]% 让文档总是结束于偶数页,可根据需要设定页眉页脚样式,如 [noheaderstyle]
%---------------------------------------------------------------------------%
