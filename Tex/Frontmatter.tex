%---------------------------------------------------------------------------%
%->> Frontmatter
%---------------------------------------------------------------------------%
%-
%-> 生成封面
%-
\maketitle% 生成中文封面
\MAKETITLE% 生成英文封面
%-
%-> 作者声明
%-
\makedeclaration% 生成声明页
%-
%-> 中文摘要
%-
\intobmk\chapter*{摘\quad 要}% 显示在书签但不显示在目录
\setcounter{page}{1}% 开始页码
\pagenumbering{Roman}% 页码符号

陆地地表风速普遍减小近年来受到广泛关注,它会影响地-气热通量、水汽通量、大气污染以及风能资源储量。为了全面研究这个现象,本文通过观测数据分析及模式模拟得出了北半球陆地地表风速长期变化的时空特征,分析了造成风速变化的两个主要原因——大气运动驱动力和大气运动阻力变化,并研究了受风速变化影响风能资源的长期变化,得到如下主要结论:

\begin{enumerate}

\item \textbf{北半球陆地地表风速长期变化的时空特征。}陆地地表风速减弱是一个普遍的现象,1979-2016年间全球有73\%的站点观测到了风速下降。在北美洲、欧洲和亚洲风速分别累积下降了-6.5\%,-9.6\%和-11.2\%。欧洲和亚洲的高百分位风速下降明显快于低百分位,而北美洲与此相反。在不同季节上,总体来看春季下降最快而夏季最慢。在不同海拔上,北美洲和欧洲海拔越高风速越趋向增加。风速趋势存在明显的年代际变化,全球平均来看风速下降发生在2010年前,之后风速趋于平稳。北美洲风速下降出现在1980至2010年间,欧洲近38年风速持续下降,亚洲风速下降发生在1990年前和1997-2007年间。在不同季节上的风速年代际变化大多与年均值一致,但欧洲夏季与其他季节差异明显,表现为2000年后显著增加。

\item \textbf{北半球陆地地表风速长期变化的影响因子。}大气运动驱动力方面1)高层风速对地表风速存在影响,但这种影响有地域差异。在欧洲和亚洲中纬度地区地表风速变化可以由高层解释,而在北美洲和亚洲则不能。2)海平面气压场同样会影响地表风速。1月份冰岛低压减弱,西伯利亚高压北部增强,阿留申低压向西北偏移,导致同期南欧和北美风速减小,而北欧和亚洲风速增加;7月份海平面气压场变化主要表现为冰岛低压增强,导致同期南欧风速增加而北欧风速减小。3)环流系统和大尺度海温也是地表风速的影响因子。TNA和NAO分别影响了北美洲和欧洲陆地地表风速,PDO和西太平洋暖池面积共同影响亚洲中纬度陆地地表风速。大气运动阻力方面1)城市扩张会使地表风速减小。遥感和实地观测表明城市化速度与风速趋势呈显著负相关关系,模式试验结果表明珠江三角洲城市变化可以解释1979年以来观测风速减小的35\%。2)植被变化对风速的影响取决于植被类型。在植被变化由林地变化主导的芬兰,NDVI与风速趋势呈显著负相关,模式试验结果表明南芬兰林地变化可以解释1979年以来观测风速减小的87\%。3)边界层湍流活动强度会影响地表风速。北美日间垂直温度递减率普遍呈负趋势,预示湍流活动减弱高层动量更少下传到地表附近,欧洲和亚洲与北美洲相反。总体来看,北美洲地表风速减弱主要可以由边界层日间垂直温度递减率减小造成的湍流混合减弱解释,欧洲主要可以由高层风速减弱和植被增加解释,而亚洲主要可以由高层风速减弱和城市化解释。

\item \textbf{北半球陆地地表风速长期变化对风能资源的影响。}受到地表风速减小的影响,风能资源也经历了明显减少的过程,2012-2016年相对1979-1983年减小了14.8\%,适于建设风电场的站点风能资源下降速度超过平均水平。亚洲风能资源减小最为明显,欧洲次之而北美洲减小最慢。气候模式历史风能资源的长期变化模拟存在缺陷,使得未来预估结果存在较大不确定性。

\end{enumerate}

本文全面分析了北半球陆地地表风速长期趋势的空间特征、百分位变化、季节特点、海拔分布特征以及地表风速趋势年代际变化的空间特征和季节特点,并分析了高层风速、海平面气压、环流系统和大尺度海温、城市化、植被变化、边界层湍流强度等6个主要影响因子的作用,最后研究了风能资源的历史变化及气候模式对其的模拟能力。本文可以帮助全面系统理解陆地地表风速普遍下降的现象,并为风能产业的发展和规划提供参考。

\keywords{陆地地表风速,长期趋势,年代际变化,风能资源}% 中文关键词
%-
%-> 英文摘要
%-
\intobmk\chapter*{Abstract}% 显示在书签但不显示在目录

The general reduction of land surface wind speeds has received widespread attention in recent years. It could have a non-negligible impact on the land-atmosphere heat flux, water vapor flux, atmospheric pollution and wind energy resource. To comprehensively study this phenomenon, using observation data and model simulations, this paper analyzed the spatiotemporal characteristics of long-term changes in land surface wind speeds over the Northern Hemisphere, explored the reasons behind in terms of changes in atmospheric driving force and dragging force, and examined the resulting changes in wind energy resources. Major findings are as follows,

\begin{enumerate}

\item  \textbf{Spatiotemporal characteristics of long-term changes in land surface wind speeds over the Northern Hemisphere.} Weakening of land surface wind speed is a common phenomenon. 73\% of the sites across the globe experienced a decrease in wind speeds during 1979-2016. In North America, Europe and Asia wind speed have cumulatively decreased by -6.5\%, -9.6\% and -11.2\%, respectively. The high percentile wind speeds in Europe and Asia have dropped significantly faster than the low percentiles, while the situation in North America is the opposite. In terms of seasonality, in general, the decline is the fastest in spring and the slowest in summer. There are also differences in wind speed trends at different altitudes. In terms of altitude, the higher the altitude in North America and Europe, the more the wind speed tends to increase. Wind speed trends have clear interdecadal characteristics. Globally, the decline in wind speed occurred before 2010, after which the wind speed stabilized. The decline in wind speed in North America occurred between 1980 and 2010, the wind speed in Europe continued declining in the past 38 years, and wind speed decline in Asia occurred between 1990 and 1997-2007. The interdecadal changes in wind speeds in different seasons are mostly consistent with the annual average. However, European summer is distinct with other seasons, showing a significant increase after 2000.

\item  \textbf{Influencing factors of long-term changes in land surface wind speeds over the Northern Hemisphere.} Atmospheric driving force 1)Lower tropospheric wind speeds have impact on surface wind speeds, but not the same in everywhere. Surface wind speed change in Europe and Asian mid-latitude area can be explained by lower troposphere wind speed change, but not in North America and Asian low-latitude area. 2)The changes in the sea level pressure field can influence surface wind speed. Sea level pressure change in January are mainly featured by the weakening of the Icelandic low pressure the strengthening of the northern Siberian high, and the northwest shift of the Aleutian low. These changes will cause the wind speeds in southern Europe and North America to decrease during the same period, while the wind speeds in Northern Europe and Asia will increase. The changes in the sea level pressure field in July are mainly manifested by the increase of the Icelandic low pressure, which will increase the wind speed in southern Europe and reduce the wind speed in Northern Europe in the same period.  3)Circulation systems and large-scale sea surface temperature are also among the influencing factors of surface wind speeds.The TNA and NAO have an impact on changes in land surface wind speeds in North America and, respectively, and the joint effect of PDO and the area of Western Pacific Warm Pool influences the land surface wind speeds in mid-latitudes in Asia. Atmospheric dragging force 1)Urbanization slows down surface winds. The remote sensing and in situ observation data shows urbanization speed has a significant negative correlation with the wind speed trends. The model experiment results show that the changes in city extension in the Pearl River Delta can explain 35\% of the reduction in observed wind speeds since 1979. 2)The impact of vegetation on surface wind speeds depends on vegetation type. In Finland where vegetation change dominated by woodland, NDVI is significantly negatively correlated with wind speed trends. The model simulation results show that the change in southern Finland's woodland can explain 87\% of the reduction in observed wind speeds since 1979. 3)Intensity of turbulent mixing within the boundary layer can also influence surface wind speed. The daytime lapse rate in North America generally shows a negative trend, which indicates that the turbulent activity is weakened and the high-level momentum is less transmitted to the surface. However, it is the opposite case in Europe and Asia. Overall, the weakening of surface wind speed in North America can be explained mainly by the weakening of the turbulent mixing caused by the decrease in the daytime lapse rate in the boundary layer. In Europe it can mainly be explained by the weakening of high-level wind speed and the increase of vegetation, while Asia it can mainly be explained by the weakening of high-level wind speed and urbanization.

\item  \textbf{The impact of long-term changes in land surface wind speeds over the Northern Hemisphere on wind energy resources.} Wind energy resources are relatively abundant in North America and Europe, but relatively scarce in Asia. Most of the present wind farms are concentrated in areas rich in wind energy resources, namely the central belt of North America, the North Sea coast of Europe, and the Asia Pacific coast. Affected by the reduction of surface wind speeds, wind energy resources experienced a significant reduction. Wind power potential in 2012-2016 decreased by 14.8\% compared with 1979-1983. Wind energy resources in Asia decreased most significantly, followed by Europe and North America. Wind energy resources of stations with over class 3 wind have fallen faster than the overall level. The global climate model simulations of long-term changes in historical wind energy resources have shortcomings, making the future projection results uncertain.

\end{enumerate}

This paper comprehensively analysed the spatial characteristics, percentile changes, seasonality, altitude distribution characteristics of long-term wind speed trends and spatial and seasonal features of interdecadal variation of land surface wind speed trends in the Northern Hemisphere, and examined the impact of 6 major influencing factors including lower troposphere wind speeds, sea level pressure, circulation systems and large-scale sea surface temperature , urbanization, vegetation changes, boundary layer turbulence intensity. Finally, the historical changes of wind energy resources and the simulation ability of global climate models were studied. This paper can help systematically understand the phenomenon of the widespread decline of land surface wind speed and provide a reference for the development and planning of the wind energy industry.

\KEYWORDS{Land surface wind speed, Long-term trend, Interdecadal change, Wind energy}% 英文关键词
%---------------------------------------------------------------------------%
