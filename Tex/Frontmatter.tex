%---------------------------------------------------------------------------%
%->> Frontmatter
%---------------------------------------------------------------------------%
%-
%-> 生成封面
%-
\maketitle% 生成中文封面
\MAKETITLE% 生成英文封面
%-
%-> 作者声明
%-
\makedeclaration% 生成声明页
%-
%-> 中文摘要
%-
\intobmk\chapter*{摘\quad 要}% 显示在书签但不显示在目录
\setcounter{page}{1}% 开始页码
\pagenumbering{Roman}% 页码符号

陆地地表风速普遍减小是近年来一个受到广泛关注的现象,它对于地气热通量、水汽通量、大气污染以及风能资源储量都会产生不可忽视的影响。本文主要利用观测数据及模式模拟分析了北半球陆地地表风速长期变化的时空特征,从大气运动驱动力和大气运动阻力变化两个方面分析了其背后的原因,并考察了由此造成的风能资源的长期变化,得到如下主要结论:

\begin{enumerate}

\item \textbf{北半球陆地地表风速长期变化的时空特征。}陆地地表风速减弱是一个普遍的现象,1979-2016年间有73\%的站点出现了风速下降,北美洲、欧洲和亚洲分别累积下降了-6.5\%,-9.6\%和-11.2\%。欧洲和亚洲高百分位风速下降明显快于低百分位,而北美洲的情况与此相反。风速趋势存在季节差异,总体来看春季下降最快而夏季最慢。不同海拔的风速趋势也存在差别,北美洲和欧洲海拔越高风速趋势越趋向于正,而亚洲海拔与风速趋势无显著相关。风速有显著的年代际变化,总体来看风速下降发生在2010年前,之后风速趋于平稳。5套再分析资料互相之间以及与观测对比在陆地地表风速长期趋势上均有较大差异,年代际变化也较为不一致,其中唯一一个同化了陆地地表风速观测的再分析资料JRA-55与观测的长期变化最为接近。

\item \textbf{北半球陆地地表风速长期变化的影响因子。}欧洲和亚洲中纬度地区对流层低层风速都出现了减小,但是下降速度不及地表;然而在亚洲低纬度地区和北美洲,对流层低层风速都出现了上升,尤其是北美洲,对流层低层风速普遍出现上升,由此可见地表风速下降不能完全由对流层低层变化来解释,尤其是在北美洲。海平面气压场变化主要体现在,1月份冰岛低压减弱,西伯利亚高压北部增强,阿留申低压向西北偏移,这些变化会造成同期南欧和北美风速减小,而北欧和亚洲风速增加;7月份海平面气压场变化主要表现为冰岛低压增强,会使得同期南欧风速增加而北欧风速减小。北美洲和欧洲陆地地表风速年代际变化分别受到TNA和NAO的显著影响,亚洲中纬度陆地地表风速受到PDO和西太平洋暖池面积的共同影响。城市化速度与风速趋势呈显著负相关关系,模式试验结果表明珠江三角洲城市变化可以解释1979年以来观测风速减小的35\%。NDVI数据表明近30多年来北半球植被普遍增加,但由于有些低矮植被(如草地)对10 m风速不能产生显著影响,NDVI与风速趋势无显著相关;然而在植被变化由林地变化主导的芬兰,NDVI与风速趋势呈显著负相关,模式试验结果表明南芬兰林地变化可以解释1979年以来观测风速减小的87\%。北美日间垂直温度递减率普遍呈负趋势,预示湍流活动减弱高层动量更少下传到地表附近;欧洲和亚洲与北美洲相反,预示湍流活动增强,高层动量更多向地表附近传递。总体来看,北美洲地表风速减弱主要可以由边界层日间垂直温度递减率减小造成的湍流混合减弱解释,欧洲主要可以由高层风速减弱和植被增加解释,而亚洲主要可以由高层风速减弱和城市化解释。

\item \textbf{北半球陆地地表风速长期变化对风能资源的影响。}风能资源在北美洲和欧洲相对丰富,而亚洲相对缺乏。目前已建成的风电场大部分集中在风能资源丰富的地区,即北美洲的中部地带,欧洲北海沿岸,亚洲太平洋沿岸地区。受到地表风速减小的影响,风能资源也经历了明显减少的过程,2012-2016年相对1979-1983年减小了14.8\%,亚洲风能资源减小最为明显,欧洲次之而北美洲减小最慢。适于建设风电场的大于3类风力站点风能资源下降速度超过整体水平。CMIP5对于历史风能资源的长期变化模拟存在缺陷,使得其未来预估结果存在较大不确定性。

\end{enumerate}

\keywords{陆地地表风速,长期趋势,年代际变化,对流层风速,海平面气压场,环流系统,大尺度海温,城市化,植被变化,风能资源}% 中文关键词
%-
%-> 英文摘要
%-
\intobmk\chapter*{Abstract}% 显示在书签但不显示在目录

The general reduction of land surface wind speeds has received widespread attention in recent years. It could have a non-negligible impact on the land-atmosphere heat flux, water vapor flux, atmospheric pollution and wind energy resource. Using observation data and model simulations, this paper analyzed the spatiotemporal characteristics of long-term changes in land surface wind speeds over the Northern Hemisphere, explored the reasons behind in terms of changes in atmospheric driving force and dragging force, and examined the resulting changes in wind energy resources. Major findings are as follows,

\begin{enumerate}

\item  \textbf{Spatiotemporal characteristics of long-term changes in land surface wind speeds over the Northern Hemisphere.} Weakening of land surface wind speed is a common phenomenon. 73\% of the sites experienced a decrease in wind speeds during 1979-2016, and North America, Europe and Asia have cumulatively decreased by -6.5\%, -9.6\% and -11.2\%, respectively. The high percentile wind speeds in Europe and Asia have dropped significantly faster than the low percentiles, while the situation in North America is the opposite. Seasonal differences exist in wind speed trends. In general, the decline is the fastest in spring and the slowest in summer. There are also differences in wind speed trends at different altitudes. The higher the altitude in North America and Europe, the more the wind speed tends to be positive, while in Asia altitude has no significant correlation with the wind speed trends. Wind speeds have significant interdecadal changes. Generally speaking, the wind speed decline occurred before 2010, after which the wind speed stabilized. The five sets of reanalysis data are quite different from each other and the observations in terms of long-term trends and interdecadal changes in land surface wind speeds. The only one of the reanalysis datasets which assimilated the land surface wind speed observation, JRA-55, is closest to the observed long-term changes.

\item  \textbf{Influencing factors of long-term changes in land surface wind speeds over the Northern Hemisphere.} Lower tropospheric wind speeds in both Europe and Asia's mid-latitudes have decreased, but the rate of decline is less than the surface. However, in North America and the low latitudes of Asia, the lower troposphere wind speeds have increased, especially in North America, where the lower troposphere wind speeds have increased in almost everywhere. Therefore, it can be concluded that the decrease in surface wind speed cannot be completely explained by changes in the lower troposphere, especially in North America. The changes in the sea level pressure field in January are mainly featured by the weakening of the Icelandic low pressure the strengthening of the northern Siberian high, and the northwest shift of the Aleutian low. These changes will cause the wind speeds in southern Europe and North America to decrease during the same period, while the wind speeds in Northern Europe and Asia will increase. The changes in the sea level pressure field in July are mainly manifested by the increase of the Icelandic low pressure, which will increase the wind speed in southern Europe and reduce the wind speed in Northern Europe in the same period. The inter-decadal changes in land surface wind speeds in North America and Europe are significantly affected by TNA and NAO, respectively, and the land surface wind speeds in mid-latitudes in Asia were jointly affected by PDO and the area of Western Pacific Warm Pool. The urbanization speed has a significant negative correlation with the wind speed trends. The model experiment results show that the changes in city extension in the Pearl River Delta can explain 35\% of the reduction in observed wind speeds since 1979. NDVI data shows that the vegetation in the Northern Hemisphere has generally increased in the past 30 years, but because some low vegetation (such as grassland) has no significant effect on the wind speed of 10 m, NDVI has no significant correlation with the wind speed trend. However NDVI is significantly negatively correlated with wind speed trends in Finland where vegetation change dominated by woodland. The model simulation results show that the change in southern Finland's woodland can explain 87\% of the reduction in observed wind speeds since 1979. The daytime lapse rate in North America generally shows a negative trend, which indicates that the turbulent activity is weakened and the high-level momentum is less transmitted to the surface. In Europe and Asia, the situation is opposite to North America, indicating that the turbulent activity is enhanced, and the high-level momentum is more transmitted near the surface. Overall, the weakening of surface wind speed in North America can be explained mainly by the weakening of the turbulent mixing caused by the decrease in the daytime lapse rate in the boundary layer. In Europe it can mainly be explained by the weakening of high-level wind speed and the increase of vegetation, while Asia it can mainly be explained by the weakening of high-level wind speed and urbanization.

\item  \textbf{The impact of long-term changes in land surface wind speeds over the Northern Hemisphere on wind energy resources.} Wind energy resources are relatively abundant in North America and Europe, but relatively scarce in Asia. Most of the present wind farms are concentrated in areas rich in wind energy resources, namely the central belt of North America, the North Sea coast of Europe, and the Asia Pacific coast. Affected by the reduction of surface wind speeds, wind energy resources experienced a significant reduction. Wind power potential in 2012-2016 decreased by 14.8\% compared with 1979-1983. Wind energy resources in Asia decreased most significantly, followed by Europe and North America. Wind energy resources of stations with over class 3 wind have fallen faster than the overall level. CMIP5 has shortcomings in the simulation of long-term changes in historical wind energy resources, making its future forecast results uncertain.

\end{enumerate}


\KEYWORDS{land surface wind speed, long-term trend, interdecadal change, tropospheric wind speed, sea level pressure, circulation system, large-scale sea surface temperature, urbanization, vegetation change, wind energy}% 英文关键词
%---------------------------------------------------------------------------%
