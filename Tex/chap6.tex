%  --------------------- chapter 6--------------------- %
\chapter{总结与展望}\label{chap:conclusion}

\section{本文主要结论}

北半球陆地地表风速是近年来一个显著的现象,前人已有许多研究探讨了这个问题,本文对这个问题涉及到的几个主要方面做了系统梳理,并在前人的研究基础上进行了一些推进。本文探讨了以下的科学问题:

\begin{enumerate}

\item 北半球陆地地表风速长期变化的时空特征

\item 北半球陆地地表风速变化的机理

\item 北半球陆地地表风速长期变化对风能资源的影响

\end{enumerate}

通过分析和探讨,本文就以上3个科学问题得到如下几点结论:

\begin{enumerate}

\item 使用观测资料(站点主要分布于北半球)的分析结果表明,陆地地表风速减弱是一个普遍的现象,1979-2016年间有73\%的站点出现了风速下降,全球中位数风速趋势为-0.081 $m ~ s^{-1}$每十年,北美洲、欧洲和亚洲分别以-0.075,-0.105和-0.075 $m ~ s^{-1}$每十年速度下降,1979-2016年分别累积下降了-6.5\%,-9.6\%和-11.2\%。欧洲和亚洲高百分位风速下降明显快于低百分位,而北美洲的情况与此相反。风速趋势存在季节差异,总体来看春季下降最快而夏季最慢。不同海拔的风速趋势也存在差别,北美洲和欧洲海拔越高风速越趋向于增加,而亚洲海拔与风速趋势无显著相关。风速趋势有显著的年代际变化,总体来看风速下降发生在2010年前,之后风速趋于平稳。北美洲风速下降出现在1980至2010年间,其他时间风速平稳;欧洲近38年风速持续下降,2000年之后下降慢于之前;亚洲风速下降发生在1990年前和1997-2007年间,其他时间风速平稳。在不同季节上的风速年代际变化大多与年均值接近,欧洲夏季与其他季节差异明显,表现为2000年后显著增加。为了评估观测资料结果的不确定性,使用了5套再分析资料与观测结果进行对比,得到的中位数风速趋势从0.022 $m ~ s^{-1}$每十年至-0.042 $m ~ s^{-1}$每十年不等,年代际变化也存在较明显的不一致性。

\item 北半球陆地地表风速变化的原因可以分为两个方面,即大气运动驱动力变化和大气运动阻力的变化。

在第一个方面,对流层低层风速变化可以一定程度上反映大气驱动力的变化,通过探空观测数据发现欧洲和亚洲中纬度地区对流层低层风速都出现了减小,但是下降速度不及地表;然而在亚洲低纬度地区和北美洲,对流层低层风速都出现了上升,尤其是北美洲,对流层低层风速普遍出现上升,由此可见欧洲和亚洲中纬度地表风速减小部分可以由对流层低层风速变化解释,而北美洲和亚洲低纬度不能。此外,海平面气压场变化同样可以反映大气运动驱动力变化,由观测结果长期趋势分析表明,1月份冰岛低压减弱,西伯利亚高压北部增强,阿留申低压向西北偏移,这些变化会造成同期南欧和北美风速减小,而北欧和亚洲风速增加;7月份海平面气压场变化主要表现为冰岛低压增强,会使得同期南欧风速增加而北欧风速减小。大尺度海温和环流系统变化同样会对大气运动驱动力产生影响,北美洲和欧洲陆地地表风速年代际变化分别受到TNA和NAO的显著影响,亚洲中纬度陆地地表风速受到PDO和西太平洋暖池面积的共同影响。

在第二个方面,城市化是大气运动阻力变化的重要原因之一,遥感和实地观测表明城市化速度与风速趋势呈显著负相关关系,模式试验结果表明珠江三角洲城市变化可以解释1979年以来观测风速减小的35\%。此外,植被变化同样会影响大气运动阻力,但对10 m风速的影响取决于植被类型,NDVI数据表明近30多年来北半球植被普遍增加,但由于有些低矮植被(如草地)对10 m风速不能产生显著影响,NDVI与风速趋势无显著相关;然而在植被变化由林地变化主导的芬兰,NDVI与风速趋势呈显著负相关,模式试验结果表明南芬兰林地变化可以解释1979年以来观测风速减小的87\%。边界层内的日间垂直温度递减率可以影响湍流强度从而影响高层动量向下传递,观测结果表明北美日间垂直温度递减率普遍呈负趋势,这解释了为何北美自由大气风速和地表风速趋势呈反向变化;欧洲和亚洲日间垂直温度递减率大部分呈现正趋势,预示湍流活动增强,高层动量更容易向地表附近传递。

总体来看,北美洲地表风速减弱主要可以由边界层日间垂直温度递减率减小造成的湍流混合减弱解释,欧洲主要可以由高层风速减弱和植被增加解释,而亚洲主要可以由高层风速减弱和城市化解释。

\item 风能资源在北美洲和欧洲相对丰富,而亚洲相对缺乏。目前已建成的风电场大部分集中在风能资源丰富的地区,即北美洲的中部地带,欧洲北海沿岸,亚洲太平洋沿岸地区。受到地表风速减小的影响,风能资源也经历了明显减少的过程,2012-2016年相对1979-1983年减小了14.8\%,适于建设风电场站点风能资源下降速度超过平均水平。亚洲风能资源减小最为明显(-18.7\%),欧洲次之(-15.2\%),北美洲减小最慢(-12.2\%)。CMIP5目前是未来风能预估的主要工具,然而其对于历史风能资源的长期变化模拟存在缺陷,因而其未来预估结果存在较大不确定性。

\end{enumerate}

\section{问题与展望}

本文的研究主要依赖于地表风速观测数据,然而由于数据质量等问题,最终用于分析的站点大多集中于北半球中纬度地区,而且即使在在北半球中纬度地区,站点覆盖也不均匀,例如东欧、北欧,中亚和西亚站点分布较少,这使得分析结果在空间上存在一定局限性;此外,由于观测站点多分布在人口较为集中的地区,只有少部分站点分布在人口稀少或没有人口居住的地区,这会影响观测资料的代表性,即观测资料所表现出的变化主要体现的是人为活动影响较大地区的变化,这对于风能资源变化的分析影响较大,因为风电场的选址主要在人迹罕至的地区,这通常是观测资料不能覆盖的区域。这是本文最大的局限。由于在地表风速长期变化方面,目再分析资料与观测资料差别较大,再分析资料之间同样有明显差异,因而难以用再分析资料代替观测进行分析。卫星遥感资料相对观测资料受不均一性影响较小,且覆盖范围更大,但目前还没有遥感算法可以获取陆地地表风速,但是存在海表风速的遥感数据(如QuikSCAT)。在这方面,未来的研究依赖于观测站点增加,再分析资料的可靠性提升或卫星遥感算法突破。

本文探讨了大尺度海温等影响大气运动驱动力的因素对于陆地地表风速等影响,但这部分研究主要在统计分析层面,没有进行模式模拟研究。在未来等研究中可以利用气候模式研究大尺度海温变化等如何影响陆地地表风速。

本文研究主要着眼于陆地地表风速,并发现其普遍减小,而其他研究指出,海表风速在近几十年来出现了上升,如何解释海洋与陆地反向的变化也是值得今后研究的问题。

本文研究发现CMIP5全球气候模式对北半球风能资源长期历史变化存在缺陷,然而这种缺陷是如何造成的,如何减小未来风能资源预估的不确定性值得今后进行研究。

\section{本文主要创新点}

\begin{enumerate}

\item 以往对于城市化影响地表风速的统计学研究多使用城市与乡村站点两类分别的风速变化进行对比的方式,本文首次利用城市变化的动态数据,计算了不同城市扩张速度下风速趋势的变化,得到城市扩张速度与风速趋势呈显著负相关,即城市扩张越快风速减小越迅速。


\item 本文首次使用真实的土地利用类型数据进行高分辨率数值模拟获得城市建设面积变化和林地变化对于地表风速的影响,得到城市扩张可以解释珠江三角洲地表风速变化的35\%,林地变化可以解释南芬兰地表风速变化的87\%。

\item 本文首次分析了近几十年来边界层日间垂直温度递减率的趋势,得到北美洲地表增温速度慢于边界层顶,从而解释了为何北美洲对流层低层与地表风速呈反向变化。

\item 本文首次评估了北半球风能资源的历史变化,得到2012-2016年相对1979-1983年减小了14.8\%,北美洲、欧洲和亚洲风能资源变化分别为-12.2\%、-15.2和-18.7\%,适于风电场建设的大于3类风力站点风能下降速度超过整体水平。同时评估了CMIP5全球气候模式对此变化的模拟能力,结果表明模式模拟能力有较大缺陷,因此使用CMIP5全球气候模式进行未来风能资源预估存在较大不确定性。

\end{enumerate}

